\begin{normalsize}
\section{Кратные корни и производная}

\begin{theorem-non}
    Пусть $K$ -- поле, $f \in K[X]$, $f \neq 0$, $a \in K$.
    Тогда $a$ -- кратный корень $f$, если и только если 
    $f(a) = f'(a) = 0$.
\end{theorem-non}
\begin{proof} $ $

    ``$\Longrightarrow$'':

    Пусть $a$ -- кратный корень. Тогда $f = (X - a)^2 h$, где $h$ --
    некоторый многочлен. Тогда $f(a) = 0$, $f' = ((X - a)^2 h)' =
    ((X - a)^2)' h + (X - a)^2 h' = 2(X - a) h + (X - a)^2 h' = 
    (X - a) \cdot (2h + (X - a) h')$. Ну и видно, что $f'(a) = 0$.

    ``$\Longleftarrow$'':

    Пусть $f(a) = f'(a) = 0$. Тогда $f = (X - a)g$. $f' = ((X - a)g)' =
    (X - a)'g + (X - a)g' = g + (X - a)g'$. Тогда из $f'(a) = 0$
    следует, что $g(a) = 0$. Получается, что $(X - a) \mid g$, и
    $(X - a)^2 \mid f$. 
\end{proof}

\follow$ $
    Пусть $K$ -- поле, $f \in K[X]$, $f \neq 0$, $a \in K$. Обозначим
    через $D$ наибольший общий делитель $f$ и $f'$. Тогда $a$ --
    кратный корень $f$, если и только если $D(a) = 0$.

\begin{proof}
    \begin{eqnarray*}
        a \text{ -- кратный корень } &\Longleftrightarrow&
        f(a) = f'(a) = 0 \\
        &\Longleftrightarrow& (X - a) \mid f \text{ и } (X - a) \mid f' \\
        &\Longleftrightarrow& (X - a) \mid D \\
        &\Longleftrightarrow& D(a) = 0
    \end{eqnarray*}
\end{proof}

\begin{conj}
    Характеристика кольца с 1.
    Если в этом кольце $K$ не существует $n : \underbrace{1 + 1 + 
    \dots + 1}_n = 0$, тогда $\operatorname{char} K = 0$, иначе 
    $\operatorname{char} K = \min \{n \mid \underbrace{1 + 1 + \dots + 1
    }_n = 0\}$.
\end{conj}

\notice В поле характеристика может быть только либо нулём, либо
простым числом.

\textbf{Пример.}

$\operatorname{char} \mathbb{Z} = 0$,
$\operatorname{char} \mathbb{Z}/m\mathbb{Z} = m$

\begin{theorem-non}
    Пусть $K$ -- поле нулевой характеристики. $f \in K[X]$, $a \in K$
    -- корень $f$ кратности $s \geqslant 2$. Тогда $a$ -- корень $f'$
    кратности в точности $s - 1$.
\end{theorem-non}
\begin{proof} $ $

    Положим $f = (X - a)^s g$. Тогда $(X - a) \nmid g$, и $g(a) \neq 0$.
    Имеем
    \begin{gather*}
        f' = \left( (X - a)^s g \right)' = \left( (X - a)^s \right)' g
        + (X - a)^s g' = s(X - a)^{s - 1} + (X - a)^s g' =
        (X - a)^{s - 1} h
    \end{gather*}
    где $h = sg + (X - a)g'$. При этом $h(a) = sg(a) \neq 0$ (здесь
    мы пользуемся тем, что $\operatorname{char} K = 0$). Следовательно,
    $(X - a) \nmid h$, из чего следует, что $a$ -- корень $f'$ кратности
    ровно $s - 1$.

\end{proof}

\end{normalsize}
\section{Корни из 1. Первообразные корни из 1}
\begin{normalsize}
  \begin{theorem-non}
    В качестве особого случая выделим $z=1$.
  \end{theorem-non}
  Согласно предыдущему утверждению существует $n$ комплексных корней из данного числа. \\

  Обозначим через $\zeta_k$ корни из единицы: \\ 

  $\zeta_k := \cos(\frac{2\pi k}{n} + i\sin(\frac{2\pi k}{n})),\ k=0, 1, \dots, n-1$ \\

  Отметим, что корни составят правильный n-угольник на комлпексной плоскости. \\

  Отметим интересный факт: увеличение аргумента на $\frac{2\pi}{n}$ в точности соответствует умножению на число с модулем единица и аргументом $\frac{2\pi}{n}$, то есть $\zeta_1$. \\
  % Расписать подробнее

  Получается, что $w_k = w_0\cdot\zeta_1^k = w_o\cdot\zeta_k$, где $w_k$ – $k$-ый корень из $1$.

  \begin{theorem-non}
    Пусть $n>1,\ z\in\mathbb{C}^*, w_0, w_1, \dots, w_{n-1}$-$все корни из z$. Тогда их сумма равна 0: $S:= \sum_{k=0}^{n-1}w_k = 0$. 
  \end{theorem-non}

  \begin{proof}
    Умножим всю сумма на $\zeta_1:=cos(\frac{2\pi}{n} + i\sin(\frac{2\pi}{n}))$: \\ 
    $\zeta_1S=\zeta_1*w_0 + \zeta_1*w_1 + \dots + \zeta_1*w_{n-1} = w_1 + w_2 + \dots + w_0$
  \end{proof} 
\end{normalsize}

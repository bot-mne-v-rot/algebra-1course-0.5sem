\section{Корни из 1. Первообразные корни из 1}
\begin{normalsize}
  \begin{theorem-non}
    В качестве особого случая выделим $z=1$.
  \end{theorem-non}
  Согласно предыдущему утверждению существует $n$ комплексных корней из данного числа. \\

  Обозначим через $\zeta_k$ корни из единицы: \\ 

  $\zeta_k := \cos(\frac{2\pi k}{n} + i\sin(\frac{2\pi k}{n})),\ k=0, 1, \dots, n-1$ \\

  Отметим, что корни составят правильный n-угольник на комплексной плоскости. \\

  Отметим интересный факт: увеличение аргумента на $\frac{2\pi}{n}$ в точности соответствует умножению на число с модулем единица и аргументом $\frac{2\pi}{n}$, то есть $\zeta_1$. \\
  % Расписать подробнее

  Получается, что $w_k = w_0\cdot\zeta_1^k = w_o\cdot\zeta_k$, где $w_k$ – $k$-ый корень из $1$.

  \begin{theorem-non}
    Пусть $n>1,\ z\in\mathbb{C}^*, w_0, w_1, \dots, w_{n-1}$-$все корни из z$. Тогда их сумма равна 0: $S:= \sum_{k=0}^{n-1}w_k = 0$. 
  \end{theorem-non}

  \begin{proof}
    Умножим всю сумма на $\zeta_1:=cos(\frac{2\pi}{n} + i\sin(\frac{2\pi}{n}))$: \\ 
    $\zeta_1S=\zeta_1*w_0 + \zeta_1*w_1 + \dots + \zeta_1*w_{n-1} = w_1 + w_2 + \dots + w_0=s$ \\
    $(\zeta_1-1)s = 0$, следовательно $s=0$, потому что $\zeta_1 \neq 1$, так как $n>1$. \\
    Стоит отметить, что можно было воспользоваться формулой геометрической прогрессии.
  \end{proof} 

  \begin{theorem-non}
    $\mu_n = \{ \zeta\in\mathbb{C} | \zeta^n = 1 \}$ – группа по умножению или подгруппа $\mathbb{C}^*$.
  \end{theorem-non}
  \begin{proof}
    Действительно, такое множество непусто. Произведение двух элементов будет элементов множества: 
    $\zeta^n = \tilde{\zeta}^n = 1 \Leftrightarrow (\zeta\tilde{\zeta})^n = 1$ \\
    Если $\zeta^n=1$, то обратный к нему $(\zeta^{-1})^n=1$ \\

    Таким образом, $\mu_n$ – подгруппа $\mathbb{C}^*$.
  \end{proof}

  \begin{conj}
    Группа $G$ называется циклческой, если $\exists g\in G: \\ 
    G=\{ g^k | g\in\mathbb{Z} \}$, то есть $G$ состоит из степеней $g$. \\
    Кратко это записывают так:
    $G=<g>$ \\
    Например: $\mathbb{Z}=<1>$ по операции сложения, то есть $1^2 = 1+1 =2$.

  \end{conj}

  \notice – $\mu_n$ – циклическая группа: \\
  $\mu_n = <\zeta_1>$ = $<\zeta_{n-1}>$

\end{normalsize}

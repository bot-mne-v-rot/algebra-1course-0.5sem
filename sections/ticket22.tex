% !TEX root = ../LinalColloc01.tex

\section{Евклидовы кольца и область главных идеалов}
\begin{normalsize}
    Область целостности называется \textit{евклидовой областью}
    (или евклидовым кольцом), если существует функция 
    $\nu: \ R \setminus \{0\} \rightarrow \Z_{\geqslant 0}$, такая, что: 
    \begin{enumerate}
        \item если $d|a, d \neq 0$, то $\nu(d) \leqslant \nu(a)$, причем равенство имеет место, если 
        и только если $d$ и $a$ ассоциированы
        \item для любых $a, b \in R \; (b \neq 0)$ существует представление $a = bq + r$,
        где либо $r = 0$, либо $\nu(r) < \nu(b)$
    \end{enumerate}
    Фигурирующая здесь функция $\nu$ называется \textit{евклидовой нормой}

    \textbf{Пример 1: }
    \textit{Кольцо $\Z$  евклидово, $\nu(a) = \abs{a}$}

    \notice - В качестве евклидовой нормы можно взять и $17\abs{a} + 3$

    \textbf{Пример 2: } 
    Для любого поля $K$ кольцо $K[X]$ евклидово, $\nu = \deg$. Вот уже 
    бесконечно много евклидовых колец в нашем распоряжении

    \textbf{Пример 3: }
    Евклидовым будет и кольцо целых гауссовых чисел $\Z[X] = \{a + bi | a,b \in \Z\};$
    В качестве евклидовой нормы можно взять квадрат модуля: $\nu(a + bi) = a^2 + b^2$
    
    \textbf{Пример 4: }
    Если $K$ - поле, то кольцо формальных степенных рядов $K[[X]]$ тоже евклидово!!!!!!!!!
    В качестве евклидовой нормы можно взять функцию порядка 
    \begin{gather*}
        \operatorname{ord}\left(\sum\limits_{i = 0}^{\infty}a_iX^i \right) = \min\{i | a_i \neq 0\}
    \end{gather*}
    Мы уже поняли, что $f \in K[[X]]^* \Longleftrightarrow \operatorname{ord} f = 0$. Отсюда получается, что любой 
    ряд порядка $d$ ассоциирован с $X^d$, и делимость в $K[[X]]$ описывается следубщим образом: если $f, g$ - два ненулевых элемента 
    ряда, то $f|g \Longleftrightarrow \operatorname{ord} \ f \leqslant \operatorname{ord} \ g$

    \textbf{Пример 5: }
    \begin{gather*}
        \Z_{(5)} = \left\{ \frac{a}{b} \mid a,b \in \Z, b \nmid 5 \right\}
    \end{gather*}
    
    \notice \textit{ Можно встретить определение евклидова кольца без указания первой аксиомы. Оно эквивалентно
    приведенному выше, но чуть менее удобно на практике}

    \notice \textit{ Любое поле формально удовлетворяет определению евклидова кольца, так как за евклидову 
    норму можно взять тождественный нуль}
    
    \begin{conj}
        Пусть есть кольцо $R$ и элемент $a \in R$. $(a) = \{ax | x \in R\}$. 
        Множесвто $(a)$ является главным идеалом кольца $R$
    \end{conj}

    \textbf{Пример не главного идеала: } 
    Берем кольцо $\Z[X]$. $I = \{f \in \Z[X] \mid \; 2| f[0]\}$ -  множество многочленов, 
    у которых свободный член четный

    \qquad Покажем, что любой идеал в евклидовом кольце $R$ является главным идеалом,
    то есть имеет вид $(c) = \{ca|a \in R\}$

    \begin{theorem-non}
        В евклидовом кольце все идеалы главные
    \end{theorem-non}
    \begin{proof}
        Пусть $R$ - произвольное евклидово кольцо, $\nu$ - соответствующая евклидова норма.
        Рассмотрим произвольный идеал $I \subset R$

        Если $I$ состоит только из $0$, то $I$ порожден нулем 

        Рассмотрим случай, когда $I \neq 0$. Выберем в $I$ 
        произвольный ненулевой элемент $c$, с минимальным значением евклидовой нормы и покажем, что $I = (c)$
        
        $(c) \subset I: c \in I \Longrightarrow (c) \subset I$

        $(c) \supset I$: Рассмотрим произвольный $a \in I$. Согласно второй аксиоме евклидовой нормы,
        либо $a$ кратно $c \ (r = 0)$, либо $a = cq + r$, где $\nu(r) < \nu(c)$
        
        $\begin{cases}
            a \in I \\
            c \in I
        \end{cases} \Longrightarrow r = a - cq \in I$

        Но это противоречит тому, что $\nu(c)$ - минимальное значение евклидовой нормы 
        на ненулевых элементах из $I$. Таким образом, возможна только ситуация, когда $a$ кратно $c \Longrightarrow a \in (c) 
        \Longrightarrow I \subset (c) \Longrightarrow I = (c)$ 
    \end{proof}
    \begin{conj}
        Области целостности, в которых все идеалы главные, называются \textit{областями главных идеалов} (ОГИ)       
    \end{conj}
\end{normalsize}
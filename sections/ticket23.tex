% !TEX root = ../LinalColloc01.tex

\section{Существование и линейное представление НОД в области главных идеалов}
\begin{normalsize}
\begin{theorem-non}
    Пусть $R$ - область главных идеалов 

    \begin{enumerate}
        \item Если $a, b \in R$, то у $a$ и $b$ существует Н.О.Д.
        \item Если $d$ - Н.О.Д. $a$ и $b$, то $d = am + bn$ для некоторых $m, n \in R$
    \end{enumerate}
    \begin{proof}
        Можно считать, что хотя-бы один из элементов $a$ и $b$ отличен от $0$.
        \begin{enumerate}
            \item Положим $I = \{am + bn | m, n \in R\} = (a, b)$ 
            
            \boxed{$Не совсем очевидно. 
                Не вполне понимаю, почему мы из головы можем так сделать$}
            
            $I$ идеал в $R \Longrightarrow \exists d \in R : I = (d)$, так как $R$ - ОГИ  

            $a, b \in I \Longrightarrow d | a, \; d | b$

            $d \in I \Longrightarrow d = am + bn, m,n \in R$

            Предположим $\begin{cases}
                d' | a \\
                d' | b
            \end{cases} \Longrightarrow d' | (am + bn) \Longrightarrow d' | d$

            \item Пусть $D$ - еще какой-либо НОД $a, b$ 

            $d | a, \; d|b \Longrightarrow d | D \Longrightarrow D$ лежит в идеале, порожденном $d$, а это то же самое, что $I \Longrightarrow D = am' + bn'$ 
        \end{enumerate}
    \end{proof}
\end{theorem-non}
\end{normalsize}
% !TEX root = ../LinalColloc01.tex

\section{Алгебраические операции, их свойства}
\begin{conj}
    Бинарной операцией на множестве $M$ называется отображение $M\times M \to M$
\end{conj}

$M=\mathbb{Z}$ - операции сложения, умножения, вычитания

$M=\mathbb{N}$ - сложение, умножение, возведение в степень

\begin{conj}
Композиция отображений. Операция композиция. $f: M \to N$

$g: N\to P$

$g\circ f: M \to P$

$(g\circ f)(m) = g(f(m))$
\end{conj}

$ $\\
\framebox[1.02\width]{Необязательный для запоминания факт, просто
он был в ДЗ. Мало ли, спросят}
\begin{theorem-non}
    Пусть $f: X \rightarrow Y$.
    \begin{enumerate}
        \item $f$ - инъекция $\Leftrightarrow f$ - обратимая слева, т.е.
        $\exists g : Y \rightarrow X \mid g \circ f = \operatorname{id}_X$
        \item $f$ - сюръекция $\Leftrightarrow f$ - обратимая справа, т.е.
        $\exists g : Y \rightarrow X \mid f \circ g = \operatorname{id}_Y$
    \end{enumerate}
\end{theorem-non}
\begin{proof} $ $
\begin{enumerate}
    \item 
    ``$\Longleftarrow$'':
    
    Пусть $g : Y \rightarrow X$ - левая обратная к $f$,
    тогда $g \circ f = \operatorname{id}_X$
    
    Возьмём $x$ и $x' \in X$. Предположим, что $f(x) = f(x')$,
    тогда:
    \[ x = \operatorname{id}_X(x) = g(f(x)) = g(f(x')) = 
    \operatorname{id}_X(x') = x' \]
    
    Таким образом, $f(x) = f(x') \Rightarrow x = x'$, 
    значит $f$ - инъекция.
    
    ``$\Longrightarrow$'':
    
    Пусть $f$ - инъекция. Возьмём $x' \in X$.
    
    Определим $g : Y \rightarrow X$ следующим образом:
    
    \[ g(y) = \begin{cases}
        x, & \mbox{если } y = f(x) \mbox{ для некоторого } x \\
        x', & \mbox{если } y \notin \operatorname{range}(f) 
    \end{cases} \]
    
    Тогда $(g \circ f)(x) = g(f(x)) = x \,\,\, \forall x \in X
    \Rightarrow g \circ x = \operatorname{id}_X \Rightarrow g$ - 
    левая обратная к $f$.
    
    \item 
    ``$\Longleftarrow$'':
    
    Пусть $g : Y \rightarrow X$ - правая обратная к $f$,
    тогда $f \circ g = \operatorname{id}_Y$
    
    Возьмём $y \in Y$.
    \[ y = \operatorname{id}_Y(y) = f(g(y)) \]
    Таким образом, любой $y \in Y$ является образом $x = g(y)
    \Rightarrow f$ - сюръекция.
    
    ``$\Longrightarrow$'':
    
    Пусть $f$ - сюръекция $\Leftrightarrow \operatorname{range}(f) = Y$
    
    Определим $g : Y \rightarrow X$
    
    \[ g(y) = \mbox{некоторому } x : f(x) = y \]
    
    Тогда $\forall y \in Y \, y = f(g(y)) = \operatorname{id}_Y(y) 
    \Rightarrow
    (f \circ g) = \operatorname{id}_Y \Rightarrow g$ - 
    правая обратная к $f$.

\end{enumerate}
\end{proof}    

\begin{conj}
    Операция $*$(или$\cdot$) на $M$ называется коммутативной, если $\forall m_1, m_2 \in M : m_1\cdot m_2 = m_2\cdot m_1$
\end{conj}
\begin{conj}
    Операция $\cdot$ на $M$ называется ассоциативной, если $\forall m_1, m_2, m_3 \in M: (m_1\cdot m_2)\cdot m_3 = m_1\cdot(m_2\cdot m_3)$
\end{conj}
\begin{theorem-non}
Общая ассоциативность. Если операция ассоциативна, то 
значение не зависит от расстановки скобок.
\end{theorem-non}
\begin{proof}
Индукция по $k$

База: $k=3$, обычная ассоциативность

Переход $\{1..(k-1)\} \to k$: \\
$B = (\ )\cdot(\ ) \overset{\text{ИП}}{=} (a_1a_2...a_l)(a_{l+1}...a_k) 
= (a_1a_2...a_l)((a_{l+1}...a_{k-1})a_k) = 
((a_1a_2...a_l)(a_{l+1}...a_{k-1}))a_k \overset{\text{ИП}}{=}
(a_1a_2...a_{k-1})a_k = a_1a_2...a_k$
\end{proof}
\begin{conj}
    $g \in M,\ n \in \mathbb{N}$ $g^n=\overbrace{g*g*g*...*g}^n$
\end{conj}

\begin{conj}
    $e \in M$ называется левым нейтральным, если $\forall m \in M: e\cdot m = m$, правым нейтральным, если $\forall m \in M: m\cdot e = m$, нейтральным, если является левым и правым нейтральным
\end{conj}
\begin{theorem-non}
    Существует не более одного нейтрального элемента.
\end{theorem-non}
\begin{proof}
    Пусть $e'$ и $e''$ -- нейтральные элементы.
    Тогда $e' = e'e'' = e''$
\end{proof}
\begin{conj}
    $a \in M$. Элемент $b \in M$ называется обратным к $a$, если $ab = ba = e$ (также можно разделять левый и правый обратный).
    Для обратного используется следующая запись: $a^{-1} = b$.
\end{conj}
\begin{conj}
    $m \in M$, $n \in \mathbb{Z}$. Тогда
    $m^n = \begin{cases}
    m\cdot m\cdot ...\cdot m,\ n>0\\
    e,\ n=0\\
    m^{-1}\cdot ... \cdot m^{-1},\ n<0
    \end{cases}$
\end{conj}
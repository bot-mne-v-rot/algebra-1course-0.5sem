\section{Делимость и ассоциированные элементы. Определение НОД}
\begin{normalsize}
    $R$ обозначает произвольное евклидово кольцо, $\nu$ - соответствующую евклидову норму
    \begin{theorem-non}
        Пусть есть $a, b \in R$
        \begin{conj}
            Элемент $x$ делит элемент $y \Longleftrightarrow \exists z: \; y = x \cdot z$ ($y$ лежит в главном идеале, который порожден элементом $x$)
        \end{conj}
        \begin{conj}
            Элемент $d \in R$ называется наибольшим общим делителем $a$ и $b$, если \begin{itemize}
                \item $d | a, d | b$
                \item Если $\exists d' : d' | a, d' | b$, то $d' | d$
            \end{itemize}
        \end{conj}
    \end{theorem-non}
        \begin{conj}
            Элементы $a, b \in R$ назыываются ассоциированными, если $a|b$ и $b|a$
        \end{conj}
        \textbf{Лемма.}
        \textit{Пусть $d$ - НОД $a$ и $b$, и $d'$ - элемент $R$. \\
        Тогда $d'$ тоже НОД $a$ и $b \Longleftrightarrow d'$ ассоциирован с $d$}

        \textbf{Лемма.} 
        \textit{
            Пусть $R$ - область целостности, $a, b$ - ее элементы. $a$ ассоциирован с $b \Longleftrightarrow b = a \cdot \varepsilon$, 
            где $\varepsilon$ - обратимый элемент 
        } 
        \begin{proof} \quad \\
            ``$\Longleftarrow$'': Если $\underbrace{b = a \cdot \varepsilon}_{a | b} \Longrightarrow \underbrace{a = b \cdot \varepsilon^{-1}}_{b | a}$ 

            ``$\Longrightarrow$'': $a$ ассоциирован с $b \Longrightarrow a$ и $b$ являются кратными. $b = a \cdot \varepsilon, \varepsilon \in R \\
            a = b \cdot \varepsilon', \varepsilon' \in R$. Подставим одно равенство в другое и получим 
            \begin{gather*}
                a = a\varepsilon\varepsilon' \\
                a(\varepsilon\varepsilon' - 1) = 0
            \end{gather*}
            Если $a \neq 0$, так как $R$ - ОЦ, то $\varepsilon\varepsilon' = 1 \Longrightarrow \varepsilon$ обратим 
            
            Если $a = 0$, то и $b = 0$, тогда в качестве $\varepsilon$ можно просто взять 1
        \end{proof}
\end{normalsize}
